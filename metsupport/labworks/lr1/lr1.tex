\documentclass[a4paper,report,14pt]{ncc}
%\usepackage{fontspec}
%\setmainfont{Times New Roman}
\usepackage[russian]{babel}
\usepackage[T2A]{fontenc}
\usepackage[utf8]{inputenc}
\usepackage{indentfirst}
\usepackage{textcomp}
\setcounter{secnumdepth}{1}


\begin{document}
\section*{Лабораторная работа №1 \\ Разработка программных эмуляций аппаратного обеспечения}

Цели и задачи лабораторной работы

\begin{itemize}
 \item получение навыков программирования на языке С++;
 \item изучение основных принципов построения программного обеспечения по эмуляции работы аппаратного обеспечения.
\end{itemize}


\subsection{Теоретический материал и задания для получения навыков и умений}

В качестве языка разработки программного обеспечения для дисциплины <<Аппаратные средства вычислительной техники>> был выбран язык С++. Данный язык позволяет реализовавывать практически все применяемые на сегодняшний день парадигмы программирования и при этом имеет возможность прямого доступа к аппаратному обеспечению компьютера.

Каждая программа на C++ содержит одну или несколько функций, причем одна из них должна иметь имя \textit{main}. Запуская программу С++, операционная система вызывает именно функцию \textit{main}. Примером простейшей программы на С++ является программа, представленная ниже.

\begin{verbatim}
Листинг 1.

 int main()
 {
     return 0;
 }
\end{verbatim}

Определение функции состоит из четырех элементов:

\begin{enumerate}
 \item тип возвращаемого значения;
 \item имя функции;
 \item список параметров;
 \item тело функции.
\end{enumerate}


Функция main имеет возвращаемый тип int, который является встроенным типом языка. Это связано с тем, что единственный оператор, который используется в функции main - return - используется операционной системой как индикатор результата работы программы. По сложившейся традиции код 0 обозначает успешное завершение программы, а коды отличные от нуля являются кодами ошибок.

\subsubsection{Компиляция и выполнение программы}

Написанную программу необходимо откомпилировать. Способ компиляции программы зависит от используемой операционной системы и компилятора.

В качестве оснонвной операционной системы при изучении дисциплины <<Аппаратные средства вычислительной техники>> используется Linux Mint (либо любой другой deb-based дистрибутив). Для обеспечения одинаковой компиляции программ в различных операционных системах предлагается использовать компилятор \textit{gcc}. Для того чтобы его использовать, создайте файл с именем lr1.cpp в директории \\ /home/\%username\%/\%groupname\%/student\_lastname, после чего запишите в него код, представленный в листинге 1.

Для того, чтобы скомпилировать команду необходимо выполнить следующую команду:

\begin{verbatim}
 user@host$ gcc lr1.cpp -o lr1 -lstdc++
\end{verbatim}

Данная команда выполняет следующие действия:

\begin{itemize}
 \item[-o] - выполняет создание исполняемого файла, имя которог указывается после ключа
 \item[-lstdc++] - данный ключ указывает компилятору, что программа написана именно на языке С++.
\end{itemize}

\subsubsection{Типы данных}

С++ является строго типизированным языком,поэтом тип данных указывает при инициализации переменной. Как и во всех языках, в С++ имеются предопределенные типы данных.

Арифметические типы данных бывают следующие:

\begin{center}
\begin{tabular}{|l|p{7cm}|c|}\hline
Тип & Значение & Максимальный размер\\\hline
bool & логический тип & не определен\\\hline
char & символ & 8 бит\\\hline
wchar\_t & двойной символ & 16 бит\\\hline
char16\_t & символ Unicode & 16 бит\\\hline
char32\_t & симовл Unicode & 32 бита\\\hline
short & Целое & 16 бит\\\hline
int & Целое & 16 бит\\\hline
long & Длинное целое & 32бита\\\hline
long long & Длинное целое & 64 бита\\\hline
float & Число с плавающей точкой & 6 значащих цифр\\\hline
double & Число с плавающей точкой двойной точности & 10 значащих цифр\\\hline
long double & Число с плавающей точкой повышенной точности & 10 значащих цифр\\\hline
\end{tabular}
\end{center}

За исключением типа bool, все остальные типы могут являться как знаковыми (signed), так и беззнаковыми (unsigned). Если знаковость типа не определена в тексте программы, то она определяется компилятором исходя из аппаратного обеспечения на котором запущена программа, однако в большинстве случаев арифметические типы все-таки по умолчанию являются знаковыми.

\subsubsection{Ввод и вывод данных. Структуры следования}

Ввод и вывод данных осуществляется следующим образом:

\begin{verbatim}
 Листинг 3.
 
#include <iostream>

using namespace std;

int main() {

std::cout<<"Simple output"<<"\n";

return 0;
}
\end{verbatim}

В листинге 3 появляется несколько новых определений. В начале программы подключается библиотека для работы со стандартным потоком ввода/вывода, после чего указывается директива using namespace std, которая позволяет обращаться напрямую к функциям стандартной библиотеки. Чтобы проверить это, удалите из программы в листинге 3 строку с using, после чего попробуйте скомпилировать программу. Затем верните эту строку, и удалите std:: и проверьте результат.

Ввод данных осуществляется по способу, представленному в листинге 4.

\begin{verbatim}
Листинг 4.

#include <iostream>

using namespace std;

int main() {

int x;

cin>>x;

x=x*2;

cout<<"You're doubled number="<<x<<"\n";

return 0;
}

\end{verbatim}



\subsubsection{Условный оператор и операторы цикла}

Условный оператор реализуется в С++ следующим образом:
\begin{verbatim}
Листинг 5.

#include <iostream>

using namespace std;

int main() {

int x;

cin>>x;

x=x*2;

if (x<50)
{
	cout<<"less 50";
}
else
{
	cout<<"More 50 or equal";
}

return 0;
}


\end{verbatim}

Оператор цикла с перечислением реализуется следующим образом:

\begin{verbatim}
Листинг 6.

#include <iostream>

using namespace std;

int main() {


for (int i=0;i<10;i++){
	cout<<i<<"\n";
}

return 0;
}

\end{verbatim}

Оператор цикла с предусловием реализуется следующим образом(выполняется то же самое, что и листингом выше):

\begin{verbatim}
Листинг 7.

#include <iostream>

using namespace std;

int main() {

int i=0;

while (i<10)
{
	cout<<i<<"\n";
	i++;
}

return 0;
}

\end{verbatim}


\subsubsection{Некоторые функции}

Для решения некоторых математических задач может использоваться библиотека \textit{cmath}, которая подключается аналогично библиотеке iostream. Некоторые функции этой библиотеки представлены в таблице.

\begin{tabular}{|l|p{7cm}|p{5cm}|}
\hline
abs( a ) & модуль или абсолютное значение от а & abs(-3.0)= 3.0;abs(5.0)= 5.0\\\hline
sqrt(a) & корень квадратный из а, причём а не отрицательно & sqrt(9.0)=3.0\\\hline
pow(a, b) & возведение  а в степень b & pow(2,3)=8\\\hline
ceil( a ) & округление а до наименьшего целого, но не меньше чем а & ceil(2.3)=3.0;ceil(-2.3)=-2.0\\\hline
floor(a) & округление а до наибольшего целого, но не больше чем а & floor(12.4)=12;floor(-2.9)=-3\\\hline
fmod(a, b) & вычисление остатка от  a/b & fmod(4.4, 7.5) = 4.4;fmod( 7.5, 4.4) = 3.1\\\hline
exp(a) & вычисление экспоненты $e^a$ & exp(0)=1\\\hline
sin(a) & a задаётся в радианах & \\\hline
cos(a) & a задаётся в радианах & \\\hline
\end{tabular}



\subsection{Задания для самостоятельного выполнения}

C помощью языка С++ решите следующие задачи:
\begin{enumerate}
 \item Найти площадь кольца по заданным внешнему и внутреннему радиусам.
 \item Даны катеты прямоугольного треугольника. Найти его периметр.
 \item Известны координаты на плоскости двух точек. Составить программу вычис-
ления расстояния между ними.
\item Даны основания и высота равнобедренной трапеции. Найти периметр трапе-
ции.
\item Треугольник задан координатами своих вершин. Найти периметр и площадь
треугольника.
\item Даны объемы и массы двух тел из разных материалов. Материал какого из тел
имеет большую плотность?
\item Год является високосным, если его номер кратен 4, однако из кратных 100
високосными являются лишь кратные 400, например, 1700, 1800 и 1900 — не-
високосные года, 2000 — високосный. Дано натуральное число n. Опреде-
лить, является ли високосным год с таким номером.
\item Дано пятизначное число. Найти число, получаемое при прочтении его цифр
справа налево.
\item Дано шестизначное число. Найти сумму его цифр. Величины для хранения
всех шести цифр числа не использовать.
\item Дано натуральное число. Определить номер цифры 8 в нем, считая от конца
числа. Если такой цифры нет, ответом должно быть число 0, если таких цифр
в числе несколько — должен быть определен номер самой левой из них.
\item Дано натуральное число. Определить, сколько раз в нем встречается мини-
мальная цифра (например, для числа для числа 102 200 ответ равен 3, для
числа 40 330 — 2, для числа 10 345 — 1).
\item Найти 20 первых натуральных чисел, делящихся нацело на 13 или на 17 и на-
ходящихся в интервале, левая граница которого равна 500.
\item Дано натуральное число. Если в нем есть цифры a и b, то определить, какая из
них расположена в числе правее. Если одна или обе эти цифры встречаются
в числе несколько раз, то должны быть учтены самые правые из одинако-
вых цифр.
\item Найти наибольший общий делитель двух заданных натуральных чисел, ис-
пользуя алгоритм Евклида.
\item Найти наименьшее общее кратное двух заданных натуральных чисел.
\end{enumerate}


% 
%  \subsection{Контрольные вопросы}
%  
%  \begin{enumerate}
%   \item 
%  \end{enumerate}
% 
% 
\subsection{Содержание и оформление отчета}

В отчет по лабораторной работе включается следующая информация:

\begin{enumerate}
 \item Название лабораторной работы
 \item Оснонвные положения теоретического материала
 \item Результаты выполения самостоятельных заданий в виде листингов программ.
\end{enumerate}

\end{document}